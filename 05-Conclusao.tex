% ---------------------------------------------------------------------------
\section{Conclusão}\label{sec:conclusao}

Embora em estágio incial de desenvolvimento, o \emph{Termpot} permitiu revisita questões relativas à performance, e linguagem de programação textual simplificada para músicos. Entretanto, apontamos o seguinte problema técnico: apontamos a biblioteca jQuery como uma provável fonte de \emph{xruns}\footnote{\emph{Google Chrome} 44.0.2403.89 Ubuntu 14.04.}. O conceito é original do ALSA (Linux), mas é semelhante do ponto de vista perceptivo. Segundo \cite{markc_xruns_2013}  um \emph{xrun} ``(\ldots) pode ser um estouro de buffer ou de uma saturação de um buffer. Um aplicativo de áudio ou não foi rápido o suficiente para transmitir dados (\ldots) ou não rápido o suficiente para processar os dados"\footnote{Tradução nossa de \emph{An "xrun" can be either a buffer underrun or a buffer overrun. In both cases an audio app was either not fast enough to deliver data (\ldots)  or not fast enough to process data}.}. Por outro lado, a criação musical baseada em funções matemáticas permite ao músico prototipar técnicas tradicionais de síntese, bem como explorar construções em cascata. Neste sentido, a ferramenta desenvolvida apresenta uma abordagem pedagógica para o ensino de música eletroacústica.

\subsection{Trabalhos Futuros}

\begin{inparaenum}[\itshape i)\upshape]
\item criar um servidor;
\item otimizar o emulador, talvez substituindo o Ptty ou propondo melhorias;
\item suporte para amostras pré-gravadas.
\end{inparaenum}

\footnote{Disponível em  \url{https://jahpd.github.io/termpot}.}. 

\subsection{Agradecimentos}

Os autores agradecem ao Guilherme Rafael Soares por ter apresentado as biblioteca Ptty.js, aos desenvolvedores dos projetos investigados por disponibilizarem seus códigos e a FAPEMIG por subsidiar a pesquisa.
