% ---------------------------------------------------------------------------
\section{Conclusão}\label{sec:conclusao}

O \emph{Termpot} está em estágio incial de desenvolvimento. Ao revisitarmos uma abordagem histórica, esbarramos com questões relativas à sintese sonora, performance, e linguagem de programação textual simplificada para músicos. Por outro lado, adinda existem problemas técnicos. Neste sentido, o desenvolvimento de um \emph{software} para criação musical baseada em funções matemáticas, aguarda o auxílio de contribuições, com interesse em uma abordagem pedagógica para o ensino de música eletroacústica.

\subsection{Trabalhos Futuros}

\begin{inparaenum}[\itshape i)\upshape]
\item criar um servidor;
\item otimizar o emulador, talvez substituindo o Ptty ou propondo melhorias;
\item suporte para amostras pré-gravadas.
\end{inparaenum}
\subsection{Agradecimentos}

Os autores agradecem ao Guilherme Rafael Soares e ao \emph{labMacambira} pelas sugestões, aos desenvolvedores do \emph{Wavepot} pelo código-aberto, e a FAPEMIG por subsidiar a pesquisa
